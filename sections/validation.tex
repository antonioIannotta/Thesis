\chapter{Validation}
\label{chap:validation}

\section{Introduction}

This chapter unfolds as a meticulous exploration into the performance and reliability of the Fair-by-Design workflow. To conduct a thorough assessment, the well known dataset Adult has been strategically chosen as the testing ground. The dataset, rich in socioeconomic and demographic attributes, provides a robust platform for evaluating the workflow's effectiveness across various machine learning algorithms.

The central emphasis of this validation endeavor is to ascertain not only the accuracy and predictive capabilities of the implemented models but also the extent to which the Fair-by-Design workflow succeeds in fostering fairness. The Adult dataset, carefully selected for its representativeness and complexity, allows for a nuanced examination of how different algorithms respond to the intricacies of real-world data, shedding light on the interplay between accuracy and fairness.

Throughout this chapter, the validation process unfolds systematically, encompassing diverse algorithms applied to the Adult dataset. The goal is to detect both the strengths and potential limitations of the Fair-by-Design approach, providing valuable insights for its application in scenarios where the equitable treatment of individuals and the accuracy of predictions are of paramount importance. The subsequent sections detail the experimental setup, algorithmic choices, and the rigorous evaluation metrics employed to ensure a comprehensive understanding of the Fair-by-Design workflow's performance on the chosen dataset.

\section{Objective definition}

In this section, we precisely outline the objectives guiding the Fair-by-Design workflow, focusing on effective income prediction while addressing fairness and mitigating biases. The key components include the identification of protected attributes, selection of fairness notions, and corresponding fairness metrics.

\subsection{Prediction Objective}

\begin{itemize}
    \item Goal: Develop models for accurate income prediction.
\end{itemize}

\subsection{Protected Attributes}

\begin{itemize}
    \item Attributes: "Ethnicity" and "Sex"
\end{itemize}

\subsection{Fairness Notions}

\begin{itemize}
    \item Demographic Parity:
    \begin{itemize}
        \item Objective: Ensure consistent distribution of predicted outcomes across different subgroups based on protected attributes.
    \end{itemize}
    
    \item Group Fairness:
    \begin{itemize}
        \item Objective: Guarantee equitable predictions within specific subgroups defined by protected attributes.
    \end{itemize}
\end{itemize}

\subsection{Fairness Metrics}

\begin{itemize}
    \item For Demographic Parity:
    \begin{itemize}
        \item Metric: Disparate Impact
    \end{itemize}
    
    \item For Group Fairness:
    \begin{itemize}
        \item Metric: Demographic Parity Difference
    \end{itemize}
\end{itemize}

These succinctly defined objectives, paired with chosen fairness notions and metrics, lay the groundwork for the Fair-by-Design workflow, ensuring a focused approach toward accurate predictions while fostering fairness across diverse subgroups.