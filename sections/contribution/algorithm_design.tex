\section{Algorithm Design}
\label{section: algorithm-design}

The stage of Algorithm Design within the fair-by-design workflow stands as a pivotal phase wherein the selection, adaptation, or creation of machine learning models takes place to harmonize with the specified objectives and fairness considerations. The primary objective is to craft models that not only excel in accuracy but also steadfastly adhere to ethical and fair principles. This section undertakes an in-depth exploration of the crucial considerations, strategic approaches, and the step-by-step implementation processes entailed in Algorithm Design. By navigating through these aspects, the aim is to foster the development of models that not only deliver superior performance but also champion the ideals of ethics and fairness in their predictive outcomes.

\subsection{Key Considerations}

\begin{enumerate}

    \item \emph{Fairness-Aware Model Selection:} 
    
    The selection of machine learning models assumes a pivotal role in the pursuit of fairness objectives. Fairness-aware model selection entails the deliberate consideration of models equipped with inherent mechanisms to tackle biases and advocate for equitable outcomes. Noteworthy examples of such models include adversarial networks, fairness-aware classifiers, and re-weighted learning algorithms. These models are systematically explored and evaluated for their capacity to mitigate biases and foster fairness, thereby contributing to the overarching goal of cultivating just and impartial predictive systems.
    
    \item \emph{Regularization Techniques:} 
    
    In the pursuit of fair predictions, regularization methods, exemplified by demographic parity constraints or equalized odds constraints, emerge as instrumental tools. These methodologies are strategically employed to guide machine learning models towards fairness objectives. By integrating fairness constraints into the optimization process, these regularization techniques play a pivotal role in steering the model to yield equitable outcomes across various demographic groups. The inclusion of such constraints serves as a proactive measure to counteract biases and ensure that the predictive power of the model remains impartial and just across diverse demographic contexts.

    \item \emph{Bias Detection and Mitigation:} 
    
    The integration of embedding mechanisms within the algorithm to detect and mitigate biases is of paramount importance. This entails the ongoing and systematic monitoring of the model's predictions for any indications of disparate impact. Subsequently, corrective measures are implemented in real-time to mitigate biases as they manifest. This proactive approach to bias detection and mitigation serves as a fundamental component in ensuring the model's fairness and equity. By embedding mechanisms that facilitate continuous scrutiny and timely interventions, the algorithm is fortified to deliver more impartial and just predictions, aligning with the principles of ethical and fair machine learning.

    \item \emph{Pre-processing, In-processing, or Post-processing Algorithms:} 
    
    The strategic decision regarding the selection of pre-processing, in-processing, or post-processing algorithms is intricately tied to the specific characteristics of the data and the defined fairness goals. Pre-processing algorithms, characterized by their ability to transform the dataset prior to model training, set the initial conditions for fairness considerations. In-processing algorithms, on the other hand, directly modify the learning process during model training, introducing fairness mechanisms at this crucial stage. Post-processing algorithms, as the name implies, come into play after model training, adjusting predictions to align with fairness objectives.

    When making this strategic decision, it is imperative to consider the nature of biases inherent in the data, the characteristics of the available dataset, and the desired level of fairness. Each approach—pre-processing, in-processing, or post-processing—offers distinct advantages and considerations, and the choice among them should be informed by a comprehensive understanding of the specific fairness challenges and objectives inherent to the machine learning task at hand.

\end{enumerate}

\subsection{Detailed Implementation Steps}

% Include the detailed implementation steps for each key consideration as per the previous discussion.

\subsubsection{Fairness-Aware Model Selection}

\begin{itemize}

    \item \emph{Exploration of Fairness-Aware Models:}
     
    \begin{itemize}
    
        \item In the endeavor to develop a fair-by-design machine learning model, a pivotal step involves conducting an exhaustive survey and evaluation of existing fairness-aware models tailored to the specific application domain. This meticulous process requires a thorough examination of pertinent literature and available models, taking into account their performance metrics, interpretability, and fairness considerations. By subjecting existing models to rigorous evaluation, practitioners gain valuable insights into the nuanced strengths and limitations of different approaches. This discerning survey and evaluation process empower practitioners to make informed decisions regarding the most suitable model for their particular use case. Such diligence contributes significantly to the foundational aspects of a fair-by-design workflow, ensuring that the selected model not only aligns with technical requirements but also upholds ethical considerations inherent to the application domain.
    
        \item When delving into the realm of fairness-aware models tailored for a specific application domain, it is judicious to prioritize models explicitly designed to address bias. This consideration encompasses, but is not confined to, adversarial networks or models featuring inherent fairness constraints. Adversarial networks serve the purpose of mitigating bias by introducing a secondary network that identifies and counteracts any discriminatory patterns discerned in the primary model. Conversely, models with built-in fairness constraints integrate predefined fairness metrics into the optimization process, ensuring the model adheres to fairness criteria throughout training.

        The evaluation of these specialized models yields a nuanced understanding of their efficacy in handling bias, furnishing practitioners with invaluable insights into the most suitable approach for mitigating bias within their specific context. This deliberate consideration solidifies the fair-by-design approach, underscoring the paramount importance of selecting models that actively address and rectify biases to foster equitable outcomes in the realm of machine learning applications.    
    
    \end{itemize}
    
    \item \emph{Implementation and Customization:}
    
    \begin{itemize}
    
        \item Following the selection of a fairness-aware model tailored for the specific application domain, the subsequent imperative involves its implementation within the chosen machine learning framework. This intricate process necessitates the translation of model specifications and architecture into code, meticulous configuration of essential parameters, and seamless integration into the overarching machine learning pipeline. The implementation procedure should strictly adhere to industry best practices and guidelines stipulated by the chosen framework, ensuring not only optimal performance but also seamless compatibility.

        Furthermore, developers must prioritize the interpretability of the model, recognizing that transparency is pivotal in comprehending how fairness considerations are intricately embedded within the system. The successful execution of the implementation phase represents a pivotal juncture in the fair-by-design workflow. It sets the groundwork for subsequent stages encompassing model training, rigorous evaluation, and eventual deployment, all underscored by an unwavering commitment to fostering equitable and unbiased machine learning outcomes.    
    
        \item Subsequent to the implementation of the chosen fairness-aware model, a pivotal phase entails customizing the model to harmonize with the unique characteristics of the dataset and the precise fairness objectives delineated earlier in the workflow. This customization process necessitates the fine-tuning of model parameters, adjustment of hyperparameters, and incorporation of features that aptly accommodate the nuanced intricacies present in the dataset.

        Moreover, developers may find it imperative to introduce specific fairness-oriented adjustments to the model's architecture. This ensures the model's adeptness in effectively handling and mitigating biases associated with protected attributes. Such meticulous and tailored customization stands as an indispensable step for optimizing the model's performance within the paradigm of the fair-by-design approach. Here, the overarching objective extends beyond technical excellence, encompassing the ethical imperative of delivering unbiased machine learning outcomes.    
    
    \end{itemize}

\end{itemize}

\subsubsection{Explainability and Interpretability}

\begin{itemize}

    \item \emph{Model Selection:}

    \begin{itemize}

        \item Within the fair-by-design workflow, the strategic choice of models renowned for their high explainability and interpretability holds paramount significance. Decision trees and rule-based models, celebrated for their inherent transparency, stand out as preferred choices in this context. The rationale underpinning this selection is deeply rooted in the commitment to transparency and accountability.

        Models characterized by easy interpretability empower stakeholders, including end-users and decision-makers, to comprehend the intricate decision-making processes. This transparency not only cultivates trust but also facilitates the identification and mitigation of any potential biases or fairness concerns embedded within the model. By deliberately opting for models with high explainability, the fair-by-design workflow aligns seamlessly with its overarching goal of cultivating machine learning systems that excel not only in technical robustness but also in ethical soundness, ensuring they are readily understandable by a diverse range of stakeholders.

        \item Within the fair-by-design workflow, when infusing fairness considerations into model selection, a critical aspect is to judiciously navigate the trade-offs between model complexity and interpretability. The optimal balance hinges significantly on the specific application requirements, playing a pivotal role in the decision-making process.

        Highly complex models, while potentially offering superior predictive performance, often come at the cost of diminished interpretability. Conversely, interpretable models such as decision trees or rule-based models may exhibit limitations in capturing intricate patterns present in the data. Striking the right balance becomes paramount; the fair-by-design approach is not solely concerned with ensuring fairness but also places a premium on transparency.
        
        Therefore, the strategic selection of models aligning with the interpretability needs of stakeholders while still meeting performance requirements becomes a key consideration in the design process. This deliberate and conscious decision-making process contributes significantly to the development of machine learning models that are not only fair but also understandable and accountable, embodying the principles of the fair-by-design approach.

    \end{itemize}

\end{itemize}

\subsubsection{Regularization Techniques}

\begin{itemize}

    \item \emph{Integration of Fairness Constraints:}

    \begin{itemize}

        \item The incorporation of fairness constraints mandates a nuanced comprehension of the precise fairness goals and metrics germane to the application domain. In the fair-by-design approach, there is a pronounced emphasis on the principled and systematic integration of these constraints into the machine learning model's training process. This deliberate embedding of fairness into the training regimen aims to mitigate biases and advocate for equitable outcomes.

        The overarching objective of the fair-by-design workflow extends beyond the mere delivery of accurate predictions. It is intricately woven with the commitment to ethical and fairness considerations, thereby contributing substantively to the cultivation of responsible and unbiased machine learning practices. This concerted effort reinforces the imperative of aligning machine learning models with ethical principles, ensuring they operate within a framework that prioritizes fairness and equity.

        \item Within the fair-by-design approach to machine learning, a pivotal facet involves the systematic experimentation with various regularization techniques. These methods, exemplified by demographic parity constraints or equalized odds constraints, assume a substantial role in the concerted effort to mitigate biases and champion fairness within machine learning models. Tailored to address specific fairness considerations linked to protected attributes, these techniques are meticulously designed to ascertain that the model's predictions remain uninfluenced by factors such as gender, race, or other sensitive attributes. This intentional and exploratory engagement with regularization techniques underscores the commitment to cultivating models that not only excel in predictive accuracy but also uphold the ethical imperative of fairness and equity.
    
    \end{itemize}

\end{itemize}

\subsubsection{Bias Detection and Mitigation}

\begin{itemize}
    
    \item \emph{Dynamic Bias Mitigation:}
    
    \begin{itemize}
    
        \item A proactive strategy within the fair-by-design framework involves the development of adaptive algorithms that dynamically adjust to emerging biases in the data. These algorithms are meticulously designed to engage in continuous monitoring and evaluation of the model's performance, adept at identifying any potential biases that may manifest during operational phases. The adaptive nature of these algorithms empowers them to dynamically tweak parameters or decision boundaries in response to emerging biases, thereby ensuring an ongoing commitment to fairness in real-world applications. This forward-thinking approach aligns seamlessly with the ethos of the fair-by-design framework, emphasizing not only initial fairness but also the continuous vigilance and adaptation necessary for sustained equity in machine learning outcomes.

        \item A pivotal step in the fair-by-design workflow involves the implementation of corrective measures, such as re-weighting or re-sampling, to actively mitigate biases as they are detected. As biases are identified through ongoing monitoring, these targeted corrective measures are strategically applied to address imbalances and promote fairness in the model's predictions. This iterative and adaptive approach not only acknowledges the inevitability of biases but also proactively works towards rectifying them, reinforcing the commitment to fairness and equity embedded within the fair-by-design methodology.

    \end{itemize}

\end{itemize}

\subsubsection{Human-in-the-Loop Approaches}

\begin{itemize}
    
    \item \emph{Stakeholder Collaboration:}
    
    \begin{itemize}
    
        \item Organize collaborative sessions with domain experts and stakeholders to delve into the intricacies of fairness considerations. These sessions serve as a platform for a comprehensive exploration of nuances related to fairness, leveraging the collective insights and expertise of domain specialists and key stakeholders. The aim is to foster a shared understanding of fairness challenges, identify contextual nuances, and coalesce diverse perspectives. Such collaborative endeavors contribute significantly to the fair-by-design approach, ensuring that the developed models align not only with technical requirements but also with the ethical and contextual considerations of the specific application domain.
    
        
        \item Engage stakeholders actively in both model validation and decision-making processes. This participatory approach ensures that the perspectives and insights of stakeholders are integral to the validation of the model's performance and the decision-making procedures. By involving stakeholders, including end-users and decision-makers, in these crucial stages, the fair-by-design workflow embraces transparency and inclusivity. This collaborative involvement contributes to the development of models that are not only technically sound but also align with the diverse needs and expectations of the stakeholders, fostering a sense of ownership and accountability in the overall machine learning process.
    
    \end{itemize}

\end{itemize}

\subsection{Pre-processing Algorithm Proposed}

\subsubsection{Fairness through data rebalancing}
\label{subsec:ftdr}

In this approach, the paradigm of bias mitigation takes on a unique and innovative perspective, one that prioritizes data augmentation over attribute removal. Unlike traditional approaches that center on the exclusion of specific attributes, this methodology embraces the concept of data augmentation, introducing a distinctive definition of fairness and equity within the AI system.

The essence of this approach revolves around the augmentation of the dataset by introducing new data instances that offer a more comprehensive and inclusive representation of the underlying population. This expanded dataset is designed to be more diverse, representative, and balanced, transcending the limitations of the original data and fostering a more nuanced understanding of fairness. 

The introduction of augmented data instances leads to a redefined notion of fairness within the AI system. Instead of solely focusing on the absence of biased attributes, fairness is now measured in terms of the dataset's inclusivity and its ability to capture the diversity and nuances present within the population it seeks to serve. 

This approach aligns with the broader philosophy of ensuring that AI systems are equitable, just, and capable of making informed and unbiased decisions. By augmenting the dataset, it strives to bridge the gaps in representation and provide a more equitable playing field for all individuals, regardless of their background or characteristics. 

The process of data augmentation necessitates a careful selection of techniques and methodologies that can introduce new data instances while maintaining the integrity and quality of the dataset. These techniques may encompass oversampling, synthetic data generation, or other data synthesis methods, each tailored to the specific context and objectives of the AI system.

In traditional fairness definitions, the focus often revolves around ensuring fair treatment for individual protected attributes, denoted as $A_1, A_2, \ldots, A_k$. While this is undoubtedly crucial, a more comprehensive understanding of fairness calls for an examination of fairness in the context of combinations of protected attributes and the output. A new definition of fairness is proposed, which takes into account the representation of all combinations of $k$ protected attributes and the output, aiming for equitable representation across these combinations.

A fair dataset is defined as one in which, for each combination of protected attributes $\{A_1, A_2, \ldots, A_k\}$ and the output $O_j$, the representation is equal and proportional. Mathematically, a dataset is fair if:

\[
\forall i_1, i_2, \ldots, i_k, j: \frac{|D_{i_1, i_2, \ldots, i_k, j}|}{|D|} = \text{constant}
\]

where:
- $D$ is the dataset,
- $|D_{i_1, i_2, \ldots, i_k, j}|$ is the number of samples with the specific combination of protected attributes $A_{i_1}, A_{i_2}, \ldots, A_{i_k}$ and output $O_j$,
- $|D|$ is the total number of samples in the dataset.

This entails that any combination of demographic groups, defined by the protected attributes, and the output should have comparable representation, thereby fostering a balanced and unbiased dataset.

By striving for equal representation of combinations of protected attributes, is addressed a fundamental aspect of fairness that transcends individual attributes. This approach provides a more nuanced understanding of fairness by considering the intersections of various demographic groups. It encourages a broader examination of potential biases that may arise when considering multiple attributes simultaneously.

Incorporating this definition of fairness into the dataset rebalancing process enables us to promote a comprehensive notion of fairness, aligning with the principles of equal opportunity and non-discrimination across all combinations of protected attributes. The subsequent algorithm and experimental evaluation are designed to actualize this definition and demonstrate its effectiveness in achieving a more equitable representation within the dataset.

At this point it's necessary to begin with a formalization for the algorithm itself.


Let \( D \) be a dataset \( R^{n \times m} \), where \( n \) is the number of samples and \( m \) is the number of features. Let \( k \) be the number of protected variables represented as \( R^{n \times 1} \), and let there be a single output variable represented as \( R^{n \times 1} \).

A rebalancing function \( \mathcal{R} \) can be formally defined as a mapping:

\[
\mathcal{R}: R^{n \times m} \rightarrow R^{l \times m}
\]

where \( l > m \), and the function \( \mathcal{R} \) transforms the input dataset \( D \) of dimensions \( n \times m \) into an output dataset \( D' \) of dimensions \( l \times m \).



Let \( k \) be the number of binary protected variables in the dataset \( D \), and consider the output variable to be binary as well. The number of possible combinations of these variables is \( 2^{(k+1)} \).

Consider a set \( \text{Combination-frequency} \) with occurrences of all \( 2^{(k+1)} \) combinations within the dataset. For each combination, the number of rows in which that combination appears should be equal to the maximum occurrence among all combinations present in the set \( \text{Combination\textunderscore frequency} \). This maximum value is denoted as \( \text{Max}(\text{Combination-frequency}) \).

Mathematically, the number of rows (\( l \)) the final dataset should have for each combination is given by:

\[
l = \text{Max}(\text{Combination-frequency})
\]



Let \( l \) be the desired number of rows for the final dataset. For each combination of values, is calculated the occurrence count \( \text{occurrence}_i \), where \( i \) ranges from 1 to \( 2^{(k+1)} \), with \( k \) being the number of protected binary variables and considering the output variable as binary.

The total number of rows to be added is given by:

\[
\text{total\_rows\_to\_add} = l - \sum_{i=1}^{2^{(k+1)}} \text{occurrence}_i
\]

For each iteration:

\begin{itemize}

    \item The values of the protected and output variables are set according to the specific combination.
    
    \item For all other attributes, a random value \( \text{random\_value}_{ij} \) is generated, where \( j \) represents the specific attribute and \( i \) represents the row being added for that attribute. \( \text{random\_value}_{ij} \) is within the minimum and maximum range for attribute \( j \).

\end{itemize}

As showed in the following pseudocode

\begin{algorithm}[H]
    \caption{Reabalancing}
    \begin{algorithmic}[1]
        \State \textbf{Input:} combination\_set, combination\_frequency, protected\_attributes, dataset\_attributes
        \State \textbf{Output:} Updated dataset

        \State max\_frequency $\gets$ max(combination\_frequency)

        \For{index \textbf{in} (0, len(combination\_set) - 1)}
            \State combination $\gets$ combination\_set[index]
            \State frequency $\gets$ combination\_frequency[index]
            \State combination\_dataset $\gets$ dataset[dataset[combination] == combination\_set[index]]

            \While{frequency $<$ max\_frequency}
                \State new\_row $\gets$ empty

                \For{(attr, val) \textbf{in} (combination, protected\_attributes)}
                    \State new\_row[attr] $\gets$ val
                \EndFor

                \For{attr \textbf{in} dataset\_attributes \textbf{and not in} protected\_attributes}
                    \State new\_row[attr] $\gets$ random(min(combination\_dataset[attr]), max(combination\_dataset[attr]))
                \EndFor

                \State dataset.add(new\_row)
                \State frequency $+$= 1
            \EndWhile
        \EndFor

        \State \textbf{return} Updated dataset
    \end{algorithmic}
\end{algorithm}

this algorithm adds rows to the dataset until every combination have the same frequency. Fundamental for this algorithm is the way in which are generate the value for the variables not involved into the combination. These are added considering the sub-dataset of the original dataset in which the combination occurs. 


\subsection{Significance}

The Algorithm Design stage is pivotal in shaping the ethical and fair behavior of machine learning models within the fair-by-design framework. By carefully selecting, customizing, and integrating fairness-aware models, and incorporating transparency and human-in-the-loop approaches, this stage ensures that the resulting models align with ethical considerations and promote equitable outcomes.
