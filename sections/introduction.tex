In recent years, Artificial Intelligence (AI) has risen to prominence across a myriad of sectors, revolutionizing our daily lives. Its impact spans from medical diagnosis to traffic management, from finance to healthcare, and beyond. One sector undergoing significant transformation thanks to AI is education. The potential to improve access to education, personalize learning, and optimize educational resources through the use of Artificial Intelligence is on the rise.\\
This work focuses on the field of education and relies on data collected within the education system of the Canary Islands. Here, AI is emerging as a key partner in shaping the future of learning. However, with this potential for benefit comes significant challenges related to fairness and ethics in education. Decisions made by AI systems, whether they pertain to access to educational programs, resource allocation, or student performance assessment, must be undertaken fairly and without discrimination.\\
The present work seeks to address the challenge of fairness in education through the development of design and software development methodologies that place fairness at the core of the process from the earliest stages. The primary goal is to ensure that AI systems in education are designed to be fair and ethical from the outset. \\
This work not only explores the current landscape of Artificial Intelligence in education but also proposes various methodologies through which fairness can be integrated from the design phase. These methodologies will provide practical guidelines for developers and key stakeholders in the education sector to ensure equitable access and equal opportunities for students.\\
The work is organized into four key chapters. "State of the Art" will provide an overview of the current applications of AI in education. "Contribution" will present the proposed methodologies for fairness in education. "Validation" will examine the practical application of these methodologies using real data from the education system in the Canary Islands. Finally, "Conclusion" will wrap up the work by summarizing the findings and outlining the implications and future directions in the context of education and Artificial Intelligence.
