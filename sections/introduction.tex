\chapter{Introduction}
\label{chap:introduction}

Artificial Intelligence (AI) has experienced an unprecedented surge in prominence and utility in recent years, emerging as a transformative force across diverse domains. From powering autonomous vehicles to aiding healthcare diagnosis and recommendation systems, AI applications have become increasingly woven into the fabric of daily lives. However, this rapid proliferation has ushered in a pressing concern about the pervasive presence of bias within AI systems.

The concept of bias in AI pertains to the inadvertent or systematic preference shown towards specific groups or characteristics within the data, algorithms, or decision-making processes. This partiality leads to outcomes that are unjust, unfair, and unequal. In this era of AI-driven decision-making, the imperative to address bias is not just a technological challenge but a moral and societal necessity. Additionally, the ethical principle of fairness underscores the collective aspiration to ensure that AI systems yield equitable and just results for all individuals, regardless of their personal attributes.

The repercussions of bias and unfairness in AI systems extend far beyond mere technological concerns. These issues carry profound societal implications, impacting vital areas such as employment, education, and access to critical services. Biased AI systems perpetuate and exacerbate existing inequalities, inadvertently reinforcing harmful stereotypes and undermining the foundational principles of justice and equality.

In the realm of education, data-driven decision-making has gained significant ground, with educational institutions increasingly relying on AI systems for tasks ranging from student admissions to evaluating learning outcomes and allocating educational resources. The stakes in this domain are notably high. Ensuring that these AI-driven education systems mitigate bias and prioritize fairness is not just a technological endeavor; it is a moral and societal imperative.

The Fair-by-Design workflow presented extends the traditional machine learning workflow by explicitly incorporating fairness considerations from the outset of AI system design. This approach aims to proactively address and mitigate bias throughout the development process, ensuring fairness is not an afterthought but an integral part of the system's foundation.

This work proposes the Fair-by-Design workflow, offering multiple solutions to address fairness challenges within AI systems. There will be explored and implemented three distinct approaches within this workflow, each contributing to the overarching goal of fostering fairness in AI.

The objective is to compare not only the accuracy but also the value of specific fairness metrics at the conclusion of the workflow. By contrasting these outcomes with a scenario where fairness assumptions are not made, the goal is to provide a comprehensive assessment of both accuracy and fairness within the proposed framework.

The structure of this thesis unfolds as follows: The \cref{chap:background} conducts a thorough review of the existing approaches and methodologies designed to address bias and promote fairness in the development of AI systems. This chapter lays the groundwork upon which the innovative Fair-by-Design workflow is built. The \cref{chap:contribution} chapter delves into the intricacies of the workflow, elucidating the seamless integration of various fairness approaches into the design process. The \cref{chap:validation} meticulously presents the results derived from the application of these fairness approaches, offering an empirical comparison of their performance and effectiveness. Finally, the \cref{chap:conclusions} not only imparts insights gleaned from this research but also outlines prospective directions for further advancements in this critical and ever-evolving field.
