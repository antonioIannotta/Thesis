%----------------------------------------------------------------------------------------
\chapter{Conclusion}
\label{chap:conclusions}
%----------------------------------------------------------------------------------------

In culmination, this thesis has provided a comprehensive exploration into the application of a fair-by-design workflow within a singular real-world scenario, employing various algorithms. The key takeaway lies in the revelation that the consistent application of the fair-by-design ethos can yield disparate outcomes, ranging from remarkable to suboptimal, contingent upon the algorithms selected.

The findings underscore the critical importance of embedding fairness considerations at every stage of the workflow. Notably, we have witnessed instances where meticulous attention to fairness from preprocessing to model deployment resulted in astonishingly positive outcomes. Conversely, the lack of diligence in incorporating fairness considerations throughout the workflow led to results that fell significantly short of equitable and ethical standards.

The implications of these divergent outcomes extend beyond the scope of this thesis. They prompt further exploration and analysis along two promising trajectories for future research. Firstly, a deep dive into the pre-processing algorithms that yielded exceptional results offers an avenue for uncovering specific methodologies that effectively mitigate biases and enhance fairness. Understanding the nuances of these pre-processing techniques may illuminate strategies applicable to a broader range of scenarios.

Secondly, this work suggests an expansion of the fair-by-design workflow to include additional steps, injecting fairness and ethical concepts at multiple stages. This holistic approach may involve scrutinizing data acquisition, refining feature engineering, and scrutinizing post-processing decisions. Analyzing the impact of injecting fairness considerations into these dimensions could provide a more nuanced understanding of the interplay between algorithmic decisions and ethical outcomes.

In conclusion, this thesis not only contributes insights into the fair-by-design paradigm but also beckons further exploration. It serves as a clarion call for ongoing research endeavors to unravel the intricacies of fairness in algorithmic decision-making. As technology advances, the ethical imperative to understand, implement, and refine fair-by-design approaches becomes increasingly paramount.

